\chapter{Summary and Outlook} \label{cha:summary}

The main goal of this thesis was to find a way to include a priori domain knowledge into the fitting process of a data-driven modeling approach. An overview of the related literature regarding the topics of linear models, model selection and B-splines was given in~\pref{cha:fundamentals}. In~\pref{cha:practical-considerations}, we have presented an algorithm to incorporate a priori domain knowledge in the fitting process using shape-constraint P-splines and additive models. The performance of the developed algorithm has been demonstrated by artificial problems, see~\pref{cha:practical-appl}, and two real-world examples, see~\pref{cha:practical-appl}.

With regards to future work, it would be interesting to create an algorithm that automatically finds the best possible combination of B-splines and tensor-product B-splines for a multi-dimensional problem with some a priori domain knowledge. For example, using 3 inputs, there are various combinations possible to create an additive model, e.g. a B-spline for dimension 1 and a tensor-product B-spline for dimension 2 and 3. An algorithm that automatically chooses the optimal combination with respect to some predefined criterion would help to enhance the usability of this approach. Another interesting aspect is to investigate further possible constraints that can be implemented using mapping and weighting matrices in the penalized least squares approach. An example of interest may be some kind of multi-peak constraint for B-splines. Other open topics are listed below.

\begin{itemize}
	\item Enforce a priori domain knowledge only on specific parts of the input space.
	\item Improve the available constraint, e.g. peak or valley constraint, to enhance robustness against noise.
	\item Extension of the available constraints for tensor-product B-splines.
	\item Runtime optimizations with regards to the number of B-splines for time critical applications.
	\item Extension of the algorithm to online learning. 
\end{itemize}