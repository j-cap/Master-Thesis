\chapter[Mathematical modeling]{Mathematical modeling of linear toothed belt drives}
\label{cha:model}
In this chapter, a mathematical model for linear toothed belt drives is derived. In a first step, the model is presented as a concentrated three-mass-spring-damper model. 
\section{Linear toothed belt drive}
\label{sec:belt_assumptions}
Example Equation:
\begin{align}
i_\text{g}=\frac{\omega_\text{m}}{\omega_\text{g}}\,,
\label{eq:Transmission}
\end{align}	
where
\begin{subequations}
	\begin{align}
	\omega_\text{m}=\dot{\varphi}_\text{m}=\DiffT{\ph_\text{m}}
	\label{eq:diff_phim}
	\end{align}
	and
	\begin{align}
	\omega_\text{g}=\dot{\varphi}_\text{g}=\DiffT{\ph_\text{g}}
	\end{align}	
\end{subequations}
are the total time-derivatives of $\ph_\text{m}$ and $\ph_\text{g}$, respectively. 

\subsection{Model assumptions}
Example citeation and enumeration, see also \cite{Jezernik2007,Jokinen2008,Ellis2001} and \cite[190ff]{Gurocak2015}:
\begin{itemize}
	\item The gearbox is assumed to be lossless and shows no backlash.
	\item The coupling between motor and gearbox is sufficiently stiff.
	\item The coupling between gearbox and driving pulley is supposed to be stiff.
	\item The toothed belt is assumed to be massless and the flexibility can be modeled by concentrated spring and damper elements, \cf \pref{fig:Schematic_belt}.
	\item The dynamics of the current-controlled motor is negligible, compared to the mechanical dynamics. 
\end{itemize} 

\subsection{Three-mass-spring-damper model}
\label{sec:diff_3_mass}
Example figure with tikz: 
\begin{figure}[h!tbp]
\begin{center}	
	\begin{tikzpicture}[thick]
	%init coord system:
%	\draw [help lines] (0,-2) grid (16,2);
%	\draw[-latex] (0,0) -- (1,0) node [midway,below]{x};
%	\draw[-latex] (0,0) -- (0,1) node [midway,left]{y};
%	\draw[fill=black] (0,0) circle(0.08);
	\node (A1) at (6.7,1.2){};
	
	%strich am oberen linken paar nach links weg
	\draw (A1.center) --++ (-2,0) node (A0){};
%	\draw (9.7,-1.5) --++($2*(1.4,0)$);
	
	
%	\draw (7.1,1.2) --++($(0,0)$) node (A1){};
	\draw (A1.center) --++($(0,0.6)$) node(D1){};
	\draw (A1.center) --++($(0,-0.6)$) node(K1){};
%	\draw (D1.center) --++(0.3,0);
%	\draw (K1.center) --++(0.3,0);
	
	
	%define commands for spring and damper
	\newcommand{\damper}[2]{\draw (#1,#2) --++(0.6,0);\draw[very thick] ($(#1,#2)+(0.6,-0.1)$)--++(0,0.2);\draw ($(#1,#2)+(1.4,0)$)--++(-0.5,0)--++(0,0.15)--++(-0.5,0);\draw ($(#1,#2)+(0.9,0)$)--++(0,-0.15)--++(-0.5,0);}
	\newcommand{\spring}[2]{\draw (#1,#2) --++(0.4,0)--++(0.05,0.1)--++(0.1,-0.2)--++(0.1,0.2)--++(0.1,-0.2)--++(0.1,0.2)--++(0.1,-0.2)--++(0.05,0.1)--++(0.4,0);}
	
%	\pgfextractx{2}{(P)}
	%damper

	\gettikzxy{(D1)}{\ax}{\ay}
	\damper{\ax}{\ay}
	\node (D1E) at ($(D1.center)+(1.4,0)$){};
	
	
	\gettikzxy{(K1)}{\ax}{\ay}
	\spring{\ax}{\ay}
	\node (K1E) at ($(K1.center)+(1.4,0)$){};
	\draw (K1E.center)--(D1E.center) node[midway](A2){};
	\node[above=1mm] (D1L) at ($1/2*(D1E)+1/2*(D1)$){$d_1$};
	\node[below=1mm] (K1L) at ($1/2*(K1E)+1/2*(K1)$){$c_1(\xi)$};


	%abstand zwischen d1 und masse
	\draw (A2.center)--++(1,0) node(A3){};
	\draw ($(A3.center)-(0,0.6)$) rectangle ++(1.6,1.2) node[above=1mm,midway](M){$m_\text{c}$};
	\node[circle,inner sep=0pt,fill=black,below=0mm of M,minimum size=1.5mm](F){};
	\draw[-latex] (F.center) --++(0.65,0)node[midway,below]{$f$};
	\draw ($(A3.center)+(1.6,0)$) --++(1,0)--++(0,0.6) node(D2){};
	\draw ($(D2.center)-(0,0.6)$)--++(0,-0.6) node(K2){};
	\gettikzxy{(D2)}{\ax}{\ay}
	\damper{\ax}{\ay}
	\node (D2E) at ($(D2.center)+(1.4,0)$){};
	\gettikzxy{(K2)}{\ax}{\ay}
	\spring{\ax}{\ay}
	\node (K2E) at ($(K2.center)+(1.4,0)$){};
	
	\draw (K2E.center)--(D2E.center) node[midway](A5){};
	%End spring damper 2
	\draw (A5.center)--++(2,0) node(A6){};
	\node[above=1mm] (D2L) at ($1/2*(D2E)+1/2*(D2)$){$d_2$};
	\node[below=1mm] (K2L) at ($1/2*(K2E)+1/2*(K2)$){$c_2(\xi)$};
	
	\gettikzxy{(A6.center)}{\ax}{\ay}
	\gettikzxy{(A0.center)}{\bx}{\by}
	
%	\draw (A6) circle (0.1);
%	\draw (6.3,-1.6) circle (0.1);
%	\draw ($$) circle(0.1);
	\node(A7) at ($1/2*(\ax,0)+1/2*(\bx,0)+(0,-1.2)$){};
%	\draw ($(A7.center)-(0.5,0)$)--($(A1.center)+(-0.4,-2.8)$);
	\draw ($(A7.center)+(0.7,0)$)--(\ax,-1.2)node(A9){};
%	\draw (\ax,-1.6)--(A6.center)node[midway](B1){};
	\draw ($(A7.center)-(0.7,0)$)--++(0,0.6) node(D3){};
	\draw ($(A7.center)-(0.7,0)$)--++(0,-0.6)node(K3){};
	\gettikzxy{($(A0.center)-(A7.center)+(0.7,0)$)}{\cx}{\cy}
	\draw ($(A7.center)-(0.7,0)$)--++(\cx,0)node(A10){};
	
	
	\gettikzxy{(D3)}{\ax}{\ay}
	\damper{\ax}{\ay}
	\node (D3E) at ($(D3.center)+(1.4,0)$){};
	\node[above=1mm] (D3L) at ($1/2*(D3E)+1/2*(D3)$){$d_3$};
	\gettikzxy{(K3)}{\ax}{\ay}
	\spring{\ax}{\ay}
	\node (K3E) at ($(K3.center)+(1.4,0)$){};
	\node[below=1mm] (K3L) at ($1/2*(K3E)+1/2*(K3)$){$c_3$};
	
	\draw (K3E.center)--(D3E.center) node[midway](A8){};

	\gettikzxy{($(A0.center)-(A10.center)$)}{\ax}{\ay}
	\draw (A0) arc (90:450:1/2*\ay);
	\node[circle,inner sep=0pt,fill=black,below=0mm of M,minimum size=1mm] (P1) at ($(A0)-(\ax,1/2*\ay)$){};
	
	\gettikzxy{($(A6.center)-(A9.center)$)}{\ax}{\ay}
	\draw (A9) arc (270:630:1/2*\ay);
	\node[circle,inner sep=0pt,fill=black,below=0mm of M,minimum size=1mm] (P2) at ($(A6)-(\ax,1/2*\ay)$){};
	\draw[thin,dashed](P1.center)--++(0,1.5)node(P11){};
	\draw[thin](P11.center)--++(0,1.5)node(P111){};
	
	\gettikzxy{(P111)}{\ax}{\ay}
	\gettikzxy{(F.center)}{\bx}{\by}
	\draw[thin,dashed](F.center)--++(0,0.8)node(F1){};
	\draw[thin](F1.center)--(\bx,\ay)node(FF1){};
	\draw[thin](FF1.center)--++(0,0.3);
	\draw[thin](P111.center)--++(0,0.3);
	\draw[thin,-latex](P111.center)--(FF1.center)node[midway,above](Xi){$\xi$};
	\draw[fill=white,color=white] ($(A3.center)+(0.1,0.25)$) rectangle ++(1.4,0.25);
	\draw($(A3.center)-(0,0.6)$) rectangle ++(1.6,1.2) node[above=1mm,midway](M2){$m_\text{c}$};
	
	\draw[latex-,thin] ($(P1)+(0:0.5)$) arc(0:270:0.5);
	\draw[thin] ($(P1)+(-0.4,0.4)$)--++(-0.8,0.8)node[near end,align= center,above left=0mm and 0mm](Z3){$\varphi_\text{1}$, $i_\text{g}\tau_\text{m}$,\\$i_\text{g}\tau_\text{mR}$};
	\draw[latex-,thin] ($(P2)+(0:0.5)$) arc(0:270:0.5) node[near end,above right=-3mm and 4mm](Z4){$\varphi_\text{2}$, $\tau_2$};
	\node[below=4mm of P1](Theta1){$i^2_\text{g}\Theta_\text{e}$};
	\node[below=4mm of P2](Theta2){$\Theta_\text{r}$};
	\end{tikzpicture}
\end{center}
\caption{Schematic equivalent three-mass-spring-damper system of the belt driven system.}
\label{fig:Schematic_belt}
\end{figure}
The segments of the belt between the driving and the driven pulley, as well as the ones between the pulleys and the cart are represented by spring and damper elements with parameters $d_1$, $d_2$, $d_3$, $c_1(\xi)$, $c_2(\xi)$ and $c_3$. It is worth mentioning that the spring stiffnesses $c_1(\xi)$ and $c_2(\xi)$ depend on the cart position $\xi$ as the length of the according belt segments, that is $l_1(\xi)$ and $l_2(\xi)$, vary with the position of the cart.

Example long equation:
\begin{align}
\begin{split}
\Theta_\text{e}\ddot{\phi}_\text{m}=&\tau_\text{m}+\tau_\text{mR}-i_\text{t}c_1(\xi)\braces{i_\text{t}\ph_\text{m}-\xi}-i_\text{t}d_1\braces{i_\text{t}\dot{\varphi}_\text{m}-\dot{\xi}}\\&-i^2_\text{t}c_3\braces{\varphi_\text{m}-i_\text{g}\ph_2}-i^2_\text{t}d_3\braces{\dot{\varphi}_\text{m}-i_\text{g}\dot{\ph}_2}\,.
\label{eq:drall_1}
\end{split}
\end{align}
\pref{fig:bode_mass_stiffness} depicts the transfer function $G_{\taum,\phim}(s)$ for different cart masses.
\begin{figure}[htbp!]
\centering
\tikzsetnextfilename{massandstiffness}
	\begin{tikzpicture}	
	\begin{groupplot}[
	group style={group size= 1 by 2, vertical sep= 0.2cm},
	height=5cm,	     		 
	width=0.98\textwidth,
	grid,
	legend columns=6, legend style={cells={anchor=west}, at={(0.5,1)}, anchor=north, draw=none, name=legend_name,draw},
	xmin = 1,
	xmax = 10000,
	xmode=log,
	xtick pos=left,
	ytick pos=left,
	ytick = {20,0,-20,-40,-60,-80},
%	try min ticks = 5,
%	restrict x to domain = 0.1:10000
%	spy using outlines={circle, magnification=6, connect spies},
	]
	
	\nextgroupplot[
	y tick label style={/pgf/number format/1000 sep=},
	ylabel = Amplitude in \si{\dB},
%	xlabel = f in \si{\Hz},
	ymin = -100,
	ymax = 20,
%	xticklabels ={0,0.1,0.2,0.3,0.4,0.5,0.57},
%	legend to name=named,
	legend to name=legend_massandstiffness,
	xticklabels,
%	y filter/.code={\pgfmathparse{10*log10(\pgfmathresult)}}
	line join=round,
	ylabel style={at={(yticklabel* cs:0.5,30pt)}},
	]
	
	\addlegendimage{empty legend}					
	\addlegendentry{\textbf{$S_{\mathscr{A}}$:}}
	
	
	\addplot[color=blue,no marks]
	table[x=f, y=abs1] {graphics/massandstiffness.txt};
	\addlegendentry{$\G \bigspace$};
	
	\addlegendimage{empty legend}					
	\addlegendentry{\textbf{$S_{\mathscr{B}}$:}}
	
	\addplot[color=red,no marks]
	table[x=f, y=abs2] {graphics/massandstiffness.txt};
	\addlegendentry{$\G \bigspace$};
	\addlegendimage{empty legend}					
	\addlegendentry{\textbf{$S_{\mathscr{C}}$:}}
	
	\addplot[color=green,no marks]
	table[x=f, y=abs3] {graphics/massandstiffness.txt};
	\addlegendentry{$\G$};
	
	
	\nextgroupplot[	
	y tick label style={/pgf/number format/1000 sep=},				
	ylabel= Phase in \si{\degree},
	xlabel= f in \si{\Hz},
	ytick ={-180,-135,-90,-45,0,45},
%	yticklabels ={-180,-135,-90,-45,0,45},
	line join=round,
	ylabel style={at={(yticklabel* cs:0.5,30pt)}},
	]
	\addplot[color=blue]
	table[x=f, y=phase1] {graphics/massandstiffness.txt};
	\addplot[color=red]
	table[x=f, y=phase2] {graphics/massandstiffness.txt};
	\addplot[color=green]
	table[x=f, y=phase3] {graphics/massandstiffness.txt};	

		
	
	
	\end{groupplot}
	
	\node (l1) at ($(group c1r2.south)!0.5!(group c1r2.south)$)
	[below, yshift=-3\pgfkeysvalueof{/pgfplots/every axis title shift}]
	{\pgfplotslegendfromname{legend_massandstiffness}};
	
	\end{tikzpicture}	
	\caption[Influence of parameter changes on the frequency response function.]{Bode diagram of the plant model $G_{\tau_\text{m},\ph_\text{m}}(s)$ for the parameter sets $S_{\mathscr{A}}$, $S_{\mathscr{B}}$ and $S_{\mathscr{C}}$ given in ...}
	\label{fig:bode_mass_stiffness}
\end{figure} 

 
For positioning purposes, however, the transfer function $G_{\tau_\text{m},\xi}(s)$ from the motor torque $\tau_\text{m}$ to the cart position $\xi$ is of particular interest. In the assumed linear case, the transfer function reads as
\begin{alignat}{3}
	\nonumber G_{\tau_\text{m},\xi}(s)=&G'_{\tau_\text{m},\ph_\text{m}}(s)G_{\ph_\text{m},\xi}(s)\\=&i_\text{t}\frac{ds+c}{\tilde{a}_{4}s^4+\tilde{a}_{3}s^3+\tilde{a}_{2}s^2+\tilde{a}_{1}s}\,,
\end{alignat} 
where the first part $G'_{\tau_\text{m},\ph_\text{m}}(s)$ denotes... 