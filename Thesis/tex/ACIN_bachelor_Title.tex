% ===================
% ||   maketitle   ||
% ===================

% \begin{textblock}{breite}(x,y)
% Objekt
% \end{textblock}
%
%Die Argumente breite und x sind Vielfache des horizontalen L�ngenparameters \TPHorizModule, y ein Vielfaches von %\TPVertModule. Stellt man die Parameter auf einen Millimeter (mm) ein, dann kann jede Position mittels eines %Lineals abgelesen werden.
%
\setlength{\TPHorizModule}{1mm}
\setlength{\TPVertModule}{1mm}



\providecommand{\ACINuniversity}{\GerEng
	{Technischen Universit\"at Wien}
	{Vienna University of Technology}
}
\providecommand{\ACINfaculty}{\GerEng
	{Fakult\"at f\"ur Elektrotechnik und Informationstechnik}
	{Faculty of Electrical Engineering and Information Technology}
}
\providecommand{\ACINinstitute}{\GerEng
	{Institut f\"ur Automatisierungs- und Regelungstechnik}
	{Automation and Control Institute}
}

\providecommand{\ACINexecuted}{\GerEng
	{Ausgef\"uhrt zum Zwecke der Erlangung des akademischen Grades eines}
	{Contucted in partial fulfillment of the requirements for the degree of a}
}

\providecommand{\ACINdipling}{\GerEng
	{Bachelor of Science (BSc)}
	{Bachelor of Science (BSc)}
}

\providecommand{\ACINsupervision}{\GerEng
	{unter der Leitung von}
	{supervised by}
}

\providecommand{\ACINhandin}{\GerEng
	{eingereicht an der}
	{submitted at the}
}

\providecommand{\ACINby}{\GerEng
	{von}
	{by}
}

\providecommand{\ACINCDS}{\GerEng
	{Gruppe f\"ur komplexe dynamische Systeme}
	{Complex Dynamical Systems Group}
}

\providecommand{\Matrikelnr}{\GerEng
	{Matrikelnummer}
	{Matriculation number}
}

\renewcommand{\maketitle}{
\pagenumbering{Alph}
\begin{titlepage}
	\sffamily
    \enlargethispage{5cm}
    \unitlength 1cm

    \newlength{\Top}
    \setlength{\Top}{2.5cm} % H�he der Mittellinie durch die Logos und den Schriftzug Bachelorarbeit

	\begin{textblock*}{\textwidth}[0.03, 0.5](\hoffset+\oddsidemargin+1in, \Top+1.3cm)
		\includegraphics[height=1.75cm]{logos/TULogo_CMYK}
	\end{textblock*}

	\begin{textblock*}{\textwidth}[0, 0.5](\hoffset+\oddsidemargin+1in, \Top+1.05cm)
		\flushright
%pdf%		\vspace{0pt-\parskip-\baselineskip}
		\includegraphics[width=4cm, keepaspectratio]{logos/AcinLogoText}
	\end{textblock*}
	\begin{textblock*}{\textwidth}(\hoffset+\oddsidemargin+1in, \Top+2.7cm)
		\noindent
		\centering
		\rule{\textwidth}{0.02cm}
	\end{textblock*}

	\begin{textblock*}{15cm}[0.5,0.5](\hoffset+\oddsidemargin+1in +0.5\textwidth, 8cm)
		\noindent
		\centering
		\LARGE
		\insertTitle
	\end{textblock*}

	\begin{textblock*}{\textwidth}[0,0](\hoffset+\oddsidemargin+1in,10.5cm) %
		\centering
		\Large\textbf{\MakeUppercase{\insertSubject}}
	\end{textblock*}

 	\begin{textblock*}{\textwidth}[0,1](\hoffset+\oddsidemargin+1in, 26.5cm)
 		\centering
 		\normalsize
 		\ACINexecuted\\
 		\vspace{0.3cm}
 		{\large\ACINdipling}\\
 		\vspace{1.5cm}
 		\ACINsupervision\\
 		\vspace{0.3cm}
 		{\large \insertSupervisor \\}
 		\vspace{1.2cm}
 		\ACINhandin\\
 		\vspace{0.3cm}
 		{\Large\textmd{\ACINuniversity}\\}
 		\vspace{0.1cm}
 		{\large\ACINfaculty\\}
 		{\large\ACINinstitute\\}
 		\vspace{1.5cm}
 		\ACINby\\
 		%\vspace{0.3cm}
 		{\large \insertAuthor\\}
 		\Matrikelnr
		\insertMatrikelnummer\\
		\insertAuthorAddress\\
 		\vspace{1.1cm}
 		\insertPlaceAndTime
 	\end{textblock*}

	\begin{textblock*}{\textwidth}(\hoffset+\oddsidemargin+1in, 27cm)
		%\rmfamily
		\centering
		\noindent
		\rule{\textwidth}{0.02cm}
		{\small \bfseries\ACINCDS\\}
		{\small A-1040 Wien, Gusshausstr. 27, Internet:
		http://www.acin.tuwien.ac.at\\} \vspace*{-0.2cm}
		\rule{\textwidth}{0.02cm}
	\end{textblock*}

	\null\newpage\global\setbox\TP@holdbox\vbox{} % Hack fuer externalize Funktion von pgfplots/tikz
%pdf%	\hspace{0mm}
%pdf%	\clearpage
\end{titlepage}


	\setcounter{page}{1}

	% Include all the preface-stuff...
	%==================================
	%\ifthenelse{\boolean{LANGaustrian}}{\selectlanguage{naustrian}}{}

 	\GerEng{
 		\ifthenelse{\boolean{LANGaustrian}}{
 			\selectlanguage{naustrian}
 		}{
 			\selectlanguage{ngerman}
 		}
 	}{
 		\selectlanguage{USenglish}
 	}


	\pagestyle{Preface}
	\pagenumbering{Roman}

	% \begin{hyphenrules}{ngerman}
		% \InputIfFileExists{pre1_Preface}{}
		% {
			% \ClassError{ACIN_thesis}{File pre1_Preface does not exist!}
			% {The file pre1_Preface.tex must exist!} 
		% }
	% \end{hyphenrules}

	\begin{hyphenrules}{USenglish}
		\InputIfFileExists{pre2_Abstract}{}
		{
			\ClassError{ACIN_thesis}{File pre2_Abstract does not exist!}
			{The file pre2_Abstract.tex must exist!}
		}
	\end{hyphenrules}

	\begin{hyphenrules}{ngerman}
		\InputIfFileExists{pre3_Kurzzusammenfassung}{}
		{
			\ifthenelse{\boolean{LANGaustrian}}{
	 			\selectlanguage{naustrian}
	 		}{
	 			\selectlanguage{ngerman}
	 		}
			\ClassError{ACIN_thesis}{File pre3_Kurzzusammenfassung does not exist!}
			{The file pre3_Kurzzusammenfassung.tex must exist!}
		}
	\end{hyphenrules}

 	\GerEng{
 		\ifthenelse{\boolean{LANGaustrian}}{
 			\selectlanguage{naustrian}
 		}{
 			\selectlanguage{ngerman}
 		}
 	}{
 		\selectlanguage{USenglish}
 	}

	% Create table of contents, figures...
	%======================================

	\ifdefined\NoTableOfContents
	\else
		\tableofcontents
	\fi
	\ifdefined\NoListOfFigures
	\else
		\listoffigures
	\fi
	\ifdefined\NoListOfTables
	\else
		\listoftables
	\fi

	\clearpage
	\pagenumbering{arabic}
	\pagestyle{scrheadings}

}



\AtEndDocument{
	% Bibliography...
	%=================
%	\GerEng{
%		\bibliographystyle{IEEEtranS_de}
%	}{
%		\bibliographystyle{IEEEtranS}
%	}
%	\bibliography{bibliography}

 	% Eidesstattliche Erkl�rung
 	%===========================
 	% Eidesstattliche Erkl�rung ohne Seitennummer am Ende des Dokumentes.
 	\addchap*{Eidesstattliche Erkl�rung}
 	\thispagestyle{empty}
 
		\noindent Hiermit erkl�re ich, dass die vorliegende Arbeit ohne unzul�ssige Hilfe Dritter und ohne Benutzung anderer als der angegebenen Hilfsmittel angefertigt wurde. Die aus anderen Quellen oder indirekt �bernommenen Daten und Konzepte sind unter Angabe der Quelle gekennzeichnet.
Die Arbeit wurde bisher weder im In� noch im Ausland in gleicher oder in �hnlicher Form in anderen Pr�fungsverfahren vorgelegt.\vspace{3cm}

 	\noindent%
 	\insertPlaceAndTime
 
 	\begin{flushright}
 	\rule{5cm}{0.01cm}\\
	\insertAuthor
 	\end{flushright}

}
