% Macros zum Setzen von Formeln
%-------------------------------

\newcommand{\gradient}[1]{\left(\nabla #1 \right)}
\newcommand{\hesse}[1]{\left(\nabla^2 #1 \right)}
 
% transponiert
\newcommand{\transpose}[1]{#1^\mathrm{T}}
% Exponentialfunktion
\newcommand{\e}{\mathrm{e}}
% Imagin�re Einheit
\newcommand{\I}{\mathrm{I}}
% Einheitsmatrix E
\newcommand{\II}{\vec{E}} 
% Ableitungen
\newcommand{\dd}{\mathop{}\!\mathrm{d}}
\newcommand{\Diff}[2]{\frac{\dd#1}{\dd#2}}
\newcommand{\DiffT}[1]{\Diff{#1}{t}}
\newcommand{\DDiff}[2]{\frac{\dd^2#1}{\dd#2^2}}
\newcommand{\DDiffT}[1]{\DDiff{#1}{t}}
\newcommand{\PartDiff}[2]{\frac{\partial #1}{\partial #2}}
\newcommand{\PartDiffT}[1]{\Diff{#1}{t}}
\newcommand{\PartDDiff}[2]{\frac{\partial^2 #1}{\partial #2^2}}
\newcommand{\PartDDiffT}[1]{\DDiff{#1}{t}}

\newcommand{\lie}[1]{\mathrm{L}_{#1}}
\newcommand{\ad}[1]{\mathrm{ad}_{#1}}

% Betrag und Norm
\newcommand{\abs}[1]{\left\vert#1\right\vert}
\newcommand{\norm}[1]{\left\Vert#1\right\Vert}

% Macro f�r die Abst�nde in Gleichungen mit Nebenbedingungen
\providecommand{\with}{\,, & \qquad}
% Satzzeichen nach Formeln
\providecommand{\FullStop}{\text{~\@.\xspace}}
\providecommand{\Comma}{\text{~,\xspace}}

% Klammern
\providecommand{\of}[1]{\left(#1\right)}
\newcommand{\braces}[1]{\left(#1\right)}

%varphi
\providecommand{\ph}{\varphi}
\let\phi\varphi

%prettyref
\providecommand{\pref}[1]{\prettyref{#1}}

% Zahlenmengen
% \IfSlidesElse{
% 	\newcommand{\numset}[1]{\mathbb{#1}}
% }{
	\newcommand{\numset}[1]{\mathbbm{#1}}
% }

\newcommand{\eps}{\varepsilon}

% Operatoren
\DeclareMathOperator{\sign}{sign}
\DeclareMathOperator{\rang}{rang}
\DeclareMathOperator{\Real}{Re}
\DeclareMathOperator{\Imag}{Im}
\DeclareMathOperator{\grad}{grad}
\DeclareMathOperator{\adj}{adj}
\DeclareMathOperator{\Span}{span}
\DeclareMathOperator{\asin}{asin}
\DeclareMathOperator{\acos}{acos}
\DeclareMathOperator{\atan}{atan}
\DeclareMathOperator{\asinh}{asinh}
\DeclareMathOperator{\acosh}{acosh}
\DeclareMathOperator{\atanh}{atanh}

\newcommand{\diag}{\operatorname*{diag}}
\renewcommand{\ker}{\operatorname*{Kern}}
\newcommand{\bild}{\operatorname*{Bild}}
\newcommand{\konst}{\operatorname*{konst.}}
\newcommand{\const}{\operatorname*{const.}}

% Makro f�r Vektoren (unterscheide griechische Buchstaben)
\DeclareRobustCommand{\vec}[1]{ 				
	\ifthenelse{\equal{#1}{\omega} \OR \equal{#1}{\varphi} \OR \equal{#1}{\alpha} \OR \equal{#1}{\beta} \OR \equal{#1}{\chi} \OR \equal{#1}{\delta} \OR \equal{#1}{\varepsilon} \OR \equal{#1}{\phi} \OR \equal{#1}{\epsilon} \OR \equal{#1}{\gamma} \OR \equal{#1}{\eta} \OR \equal{#1}{\iota} \OR \equal{#1}{\kappa} \OR \equal{#1}{\lambda} \OR \equal{#1}{\mu} \OR \equal{#1}{\nu} \OR \equal{#1}{\pi} \OR \equal{#1}{\theta} \OR \equal{#1}{\vartheta} \OR \equal{#1}{\rho} \OR \equal{#1}{\sigma} \OR \equal{#1}{\varsigma} \OR \equal{#1}{\tau} \OR \equal{#1}{\upsilon} \OR \equal{#1}{\xi} \OR \equal{#1}{\psi} \OR \equal{#1}{\zeta}}{
		% F�r griechische Kleinbuchstaben muss boldsymbol verwendet werden (deckt mathbf nicht ab)
		\boldsymbol{#1}
	}{
		% Alle anderen Symbole verwenden mathbf
		\mathbf{#1}
	}
}


%------------------------------
% Macros zur Verwendung im Text
%------------------------------

% Namen
%-------
\providecommand{\Maple}{\textsc{Maple}\xspace}
\providecommand{\Matlab}{\textsc{Matlab}\xspace}
\providecommand{\MatlabSimulink}{\textsc{Matlab/Simulink}\xspace}
\providecommand{\Doxygen}{\textsc{Doxygen}\xspace}

% +++ English
% .\@ is not treated as a full stop (important for the length of the whitespace
% afterwards. \@. is always treated as a full stop.)
\providecommand{\ie}{i.\,e.\@\xspace} 
\providecommand{\eg}{e.\,g.\@\xspace}
\providecommand{\cf}{cf.\@\xspace}

% +++ German
\providecommand{\zB}{z.\,B.\@\xspace}
\providecommand{\bzw}{bzw.\@\xspace}
\providecommand{\bspw}{bspw.\@\xspace}
\AtEndOfClass{\renewcommand{\dh}{d.\,h.\@\xspace}}
\providecommand{\Dh}{D.\,h.\@\xspace}
\providecommand{\ua}{u.\,a.\@\xspace}
\providecommand{\sog}{sog.\@\xspace}
\providecommand{\usw}{usw.\@\xspace}
\providecommand{\etc}{etc.\@\xspace}
\providecommand{\ggf}{ggf.\@\xspace}
\providecommand{\ca}{ca.\@\xspace}
\providecommand{\uU}{u.\,U.\@\xspace}
\providecommand{\vgl}{vgl.\@\xspace}

% f�r W�rter mit Bindestrich. (Setzt einen Bindestrich, an dem nicht getrennt
% werden darf, l�sst aber die Trennung im folgenden Wort zu.)
\providecommand{\hypII}[2]{#1\nobreakdash-\hspace{0pt}#2}
\providecommand{\x}[1]{\hypII{$x$}{#1}}
\providecommand{\y}[1]{\hypII{$y$}{#1}}
\providecommand{\z}[1]{\hypII{$z$}{#1}}

% % Kommando um Teile von Formeln hervorzuheben
% %---------------------------------------------
% % -> Definition der Styles in ACIN_script_colors.tex
% 
% \newcommand{\SpEmphStyle}{}
% \define@key[ACINreport]{EmphEquation}{style}[]{
% 	\renewcommand{\SpEmphStyle}{style=#1}
% }
% \newcommand{\SpEmphColor}{}
% \define@key[ACINreport]{EmphEquation}{color}[black]{
% 	\renewcommand{\SpEmphColor}{\color{#1}}
% }
% 
% \newcommand{\SpEmph}[2][]{%
% 	\renewcommand{\SpEmphStyle}{}%
% 	\setkeys[ACINreport]{EmphEquation}{#1}%
% 	\renewcommand{\SpEmphColor}{\color{black}}%
% 	\setkeys[ACINreport]{EmphEquation}{#1}%
% 	\expandafter\psframebox\expandafter[\SpEmphStyle]{\SpEmphColor\displaystyle #2}%
% }










%--- Zeilenhoehe in Tabellen -------------------------------------------------
% Mit dem Befehl \TabEqn kann eine Formel in einer Tabelle gesetzt werden
% (einfach nur die Formel in die Tabelle eingeben bringt die vertikale
% Ausrichtung irgendwie durcheinander)
% \ExtraTabEqnSpace ist der Platz, der oben und unter einer Formel eingef�gt
% wird
\AtEndOfClass{
\newcommand{\ExtraTabEqnSpace}{1ex}
\makeatletter
\newcommand*{\TabEqn}[1]{%
\begingroup
	\setbox\@tempboxa=\hbox{%
	#1%
	}%
	% Hinzufuegung von 1ex zu Hoehe (\ht)
	% und Tiefe (\dp) der Box.
	% Umweg ueber \dimen@ erforderlich,
	% da man \ht, und \dp nur etwas zuweisen,
	% aber nichts hinzufuegen kann.
	\setlength{\dimen@}{\ht\@tempboxa}%
	\addtolength{\dimen@}{\ExtraTabEqnSpace}%
	\setlength{\ht\@tempboxa}{\dimen@}%
	\setlength{\dimen@}{\dp\@tempboxa}%
	\addtolength{\dimen@}{\ExtraTabEqnSpace}%
	\setlength{\dp\@tempboxa}{\dimen@}%
	\usebox\@tempboxa
\endgroup
}
\makeatother
}


%Macros
\newcommand{\phim}{\ph_\text{m}}
\newcommand{\taum}{\tau_\text{m}}
\newcommand{\G}{G_{\tau_\text{m},\ph_\text{m}}}
\newcommand{\Gomega}{G_{\tau_\text{m},\omega_\text{m}}}
\newcommand{\mc}{m_\text{c}}
\newcommand{\dc}{d_\text{c}}
\newcommand{\dm}{d_\text{m}}
\newcommand{\Thetae}{\Theta_\text{e}}
\newcommand{\itt}{i_\text{t}}
\newcommand{\ig}{i_\text{g}}
\newcommand{\xip}{\dot{\xi}}
\newcommand{\omegam}{\omega_\text{m}}
\newcommand{\Gc}{G_\text{cc,cl}G_{\taum,\phim}G_\text{dt}}
\newcommand{\GR}{G_\text{R}}%$\frac{s}{sT_\text{R}+1}$ %Realisierung der differentiation von phim nach omegam
\newcommand{\omegar}{\omega_\text{r}}

%Kapietel Reglerentwurf
\newcommand{\Tphidphim}{T_{\ph_\text{d},\ph_\text{m}}}
\newcommand{\KPPID}{K_\text{P}^\text{PID}}
\newcommand{\KIPID}{K_\text{I}^\text{PID}}
\newcommand{\KDPID}{K_\text{D}^\text{PID}}
\newcommand{\KAPID}{K_\text{a}^\text{PID}}
\newcommand{\KVPID}{K_\text{v}^\text{PID}}

\newcommand{\PID}{\KPPID+\frac{\KIPID}{s}+s\KDPID}
\newcommand{\KPPPPI}{K_\text{PP}^\text{PPI}}
\newcommand{\KPPPI}{K_\text{P}^\text{PPI}}
\newcommand{\KIPPI}{K_\text{I}^\text{PPI}}
\newcommand{\KAPPI}{K_\text{a}^\text{PPI}}
\newcommand{\KVPPI}{K_\text{v}^\text{PPI}}
\newcommand{\PI}{\KPPPI+\frac{\KIPPI}{s}}
\newcommand{\Gdt}{G_{\text{dt}}}
\newcommand{\Ar}{A_\text{r}}
\newcommand{\phirpi}{\ph^\omega_\text{r}}
\newcommand{\phirp}{\ph^\ph_\text{r}}



\newcommand{\smallspace}{\hspace{0.2cm}}
\newcommand{\bigspace}{\hspace{0.5cm}}
\newcommand{\opac}{0.85}


